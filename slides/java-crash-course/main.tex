\documentclass{beamer}
 \usetheme{Madrid}
 
\usepackage[T1]{fontenc}
\usepackage[utf8]{inputenc}
\usepackage[english]{babel}
\usepackage{minted}
\usepackage[backend=biber,style=numeric-comp,sorting=none]{biblatex}
\usepackage{csquotes}
\usepackage{hyperref}
\bibliography{references.bib}
\beamertemplatenavigationsymbolsempty

\definecolor{lightGray}{HTML}{e5e5e5}

%Information to be included in the title page:
\title{COS 212 - Java Crash Course}
\author{Evert Geldenhuys}
\institute{University of Pretoria}
\date{12 February 2019}
 

\begin{document}
 
\frame{\titlepage}
 
\begin{frame}
\frametitle{Table of Contents}
\tableofcontents
\end{frame}

\section{Hello World}
\begin{frame}{Hello World}
    \inputminted{java}{HelloWorld.java}
    
    \textbf{Compile:}
    \mint{shell}|javac *.java|
    \textbf{Run:}
    \mint{shell}|java HelloWorld|
    \textbf{Outputs:}
    \mint{shell}|Hello, World!|
\end{frame}

\section{Makefile}
\begin{frame}{Makefile}
    \inputminted[
frame=lines,
framesep=2mm,
baselinestretch=1.2,
bgcolor=lightGray,
linenos
]{text}{Makefile}
    \textbf{Compile:}
    \mint{shell}|make|
    
    \textbf{Run:}
    \mint{shell}|make run|

\end{frame}
\section{Data Types}

\begin{frame}{Primitive Data Types}
\begin{table}[]
\begin{tabular}{|l|l|l|}
\hline
\textbf{Type} & \textbf{Size} & \textbf{Range}                                  \\ \hline
boolean       & depends       & true, false                                     \\ \hline
char          & 2 bytes       & Unicode characters                              \\ \hline
byte          & 1 byte        & {[}-128, 127{]}                                 \\ \hline
short         & 2 bytes       & {[}-32768, 32767{]}                             \\ \hline
int           & 4 bytes       & {[}-2147483648, 2147483647{]}                   \\ \hline
long          & 8 bytes       & {[}-9223372036854775808, 9223372036854775807{]} \\ \hline
float         & 4 bytes       & {[}-3.4E38, 3.4E38{]}                           \\ \hline
double        & 8 bytes       & {[}-1.7E308, 1.7E308{]}                         \\ \hline
\end{tabular}
\end{table}
\footfullcite{drozdek_2013}

\begin{itemize}
    \item Primitive Data Types are not objects, they only contain primitive values
\end{itemize}
\end{frame}

\begin{frame}{Reference Data Types}
    \begin{table}[]
\begin{tabular}{|l|l|}
\hline
\textbf{Primitive type} & \textbf{Wrapper class} \\ \hline
boolean                 & Boolean                \\ \hline
byte                    & Byte                   \\ \hline
char                    & Character              \\ \hline
float                   & Float                  \\ \hline
int                     & Integer                \\ \hline
long                    & Long                   \\ \hline
short                   & Short                  \\ \hline
double                  & Double                 \\ \hline
\end{tabular}
\end{table}
\footfullcite{oracle_boxing}

\begin{itemize}
    \item Reference Data types are Object wrappers around the Primitive data types
\end{itemize}
\end{frame}

\begin{frame}{Reference Data Types}
\begin{itemize}
    \item The Java compiler will automatically convert between primitive and reference types
\end{itemize}

\inputminted[fontsize=\footnotesize]{java}{ReferenceDataTypes.java}

\end{frame}

\begin{frame}{Strings}
    \inputminted{java}{Strings.java}
    \footfullcite{oracle_strings}
\end{frame}

\section{Operators}

\begin{frame}{Operators}
\begin{table}[]
\begin{tabular}{|l|l|}
\hline
\textbf{Operators}   & \textbf{Precedence}                                                                        \\ \hline
postfix              & expr++ expr--                                                                              \\ \hline
unary                & ++expr --expr +expr -expr $\sim$ \hspace{0.1cm} !                                                          \\ \hline
multiplicative       & * / \%                                                                                     \\ \hline
additive             & + -                                                                                        \\ \hline
relational           & \textless \hspace{0.1cm} \textgreater \hspace{0.1cm} \textless{}= \hspace{0.1cm} \textgreater{}= \hspace{0.1cm} instanceof                             \\ \hline
equality             & == !=  \\ \hline
logical AND          & \&\&                                                                                       \\ \hline
logical OR           & ||                                                                                         \\ \hline
ternary              & ? :                                                                                        \\ \hline
assignment           & = += -= *= /= \%=                                                                                    \\ \hline
\end{tabular}
\end{table}
\footfullcite{oracle_operators} (Excluding bitwise operators)
\end{frame}

\section{Decision Statements}

\begin{frame}{Decision Statements}
    \inputminted[fontsize=\footnotesize]{java}{DecisionStatements.java}
    \footfullcite{drozdek_2013}
\end{frame}

\section{Loops}
\begin{frame}{Loops}
    \inputminted{java}{Loops.java}
    \footfullcite{drozdek_2013}
\end{frame}

\section{Exception Handling}
\begin{frame}{Exception Handling}
\inputminted[]{java}{ExceptionHandling.java}    
\end{frame}

\section{Generic Types}
\begin{frame}{Generic Types}
\begin{itemize}
    \item Often you want to work with data in a generic way
    \item For example code that will work for Integers and Doubles
    \item Can take advantage of the fact that everything in Java is an Object
\end{itemize}

\inputminted[]{java}{BoxNonGeneric.java}
    \footfullcite{oracle_generic_types}
\end{frame}

\begin{frame}{Generic Types}
    \inputminted[]{java}{BoxNonGeneric.java}
    \footfullcite{oracle_generic_types}
    \begin{itemize}
        \item However, cannot determine at compile time if the Box class is used correctly
        \item Code can be written to expect an Integer and a String at a different place which may cause an error
    \end{itemize}
\end{frame}

\begin{frame}{Generic Types}
    \inputminted{java}{BoxGeneric.java}
    \footfullcite{oracle_generic_types}
\end{frame}

\begin{frame}{Generic Types}
    \inputminted{java}{BoxGenericUsage.java}
    \footfullcite{oracle_generic_types}
\end{frame}

\begin{frame}{COS 212 Resources}
    These slides and additional guides will be available at:
    \url{https://cos212.evert.io/} \\
    
    Areas not covered:
    \begin{itemize}
        \item Arrays
        \item ArrayList (Dynamic Arrays)
    \end{itemize}
\end{frame}

\begin{frame}[t,allowframebreaks]
  \frametitle{References}
  \printbibliography
 \end{frame}

\end{document}

